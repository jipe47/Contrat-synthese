\chapter{Introduction}

Il existe deux types de droit :

\begin{enumerate}
	\item le droit public, qui concerne les citoyens et l'Etat et qui comporte
	
	\begin{itemize}
		\item le droit pénal : crimes, infractions, \dots
		\item le droit administratif : fonctionnaires, \dots
		\item le droit constitutionnel : élections, régions, constitution, \dots
		\item le droit fiscal
	\end{itemize}
	
	\item le droit privé, qui concerne les rapports entre citoyens, et qui compte :
	
	\begin{itemize}
		\item le droit de la famille : mariages, divorces, \dots
		\item \textbf{le droit des contrats}
		\item le droit des sociétés
		\item le droit social
	\end{itemize}
\end{enumerate}

Le contrat est un accord d'au moins deux personnes qui ont la volonté de faire naître des effets dans le domaine du droit, c'est-à-dire créer, modifier, transmettre ou éteindre des droits.

Ces droits peuvent être

\begin{itemize}
	\item réels s'ils sont relatifs à une chose (droit de propriété, d'usufruit)
	\item de créance s'ils permettent d'obtenir d'une personne une prestation positive ou une abstention
\end{itemize}

Dans le droit des contrats, on compte 3 strates :

\dessinS{1}{.5}

La TGO est un ensemble de règles tirées du code civil et communes à tous les contrats, ou spécifiques à des contrats particuliers.