\chapter{Protection juridique du logiciel}

L'auteur d'un logiciel est une sorte de propriétaire de son logiciel, mais ici rien n'est physique : c'est de la propriété intellectuelle. Il a ce droit par équité, et aussi pour stimuler la créativité artistique et scientifique (tout le monde ne copie pas sur tout le monde).

La protection des programmes est européenne ; les autorités européennes élaborent des directives, des textes votés par les membres, et qui oblige les membres à l'adopter dans leurs lois nationales. On peut directement adopter la directive telle quelle par facilité, ou bien l'adapter (certains points sont libres d'interprétation). Ainsi, l'auteur d'un logiciel européen est protégé dans toute la communauté.

Les objectifs de la directive sont
\begin{itemize} 
	\item de protéger le programme dans l'Union Européenne et un minimum au niveau international.
	\item uniformiser la protection en Europe
	\item favoriser les systèmes ouverts et interopérables
\end{itemize}

	\section{Nature de la protection}
	
	Deux grandes options : soit par le droit des brevets (droits exclusifs de propriété, longs, coûteux et révélation des secrets d'élaboration), soit par le droit d'auteur, qui a été adopté.
	
	Les logiciels sont assimilés à des oeuvres littéraires.
	
	Pour les points non visés par la loi de protection des logiciels, renvoi vers la loi générale des droits d'auteur.
	
	\section{Étendue de la protection}
	
	Si on a déjà commencé le travail préparatoire d'un programme (organigramme, etc), celui-ci est déjà protégé, mais il faut 
	
	\begin{itemize}
		\item savoir prouver qu'on en est l'auteur ;
		\item qu'il soit original
		\item qu'il permette la réalisation d'un logiciel
	\end{itemize}
	
	Il y a protection de la forme mais pas du fond, de la forme mais pas des idées. Cela permet de stimuler la création, les idées sont en libre circulation. La forme inclut les algorithmes, la démarche, etc, toute forme d'expression d'un programme.
	
	\section{Conditions de la protection}
	
	La seule condition est que le programme soit original, c'est-à-dire une création intellectuelle propre à son auteur. Mais il n'est pas original s'il est banal, c'est-à-dire que n'importe quel informaticien s'en servira ; un effort créatif est nécessaire. Un juge s'attachera au choix de l'auteur, et qui seront différents d'un auteur à l'autre.
	
	\section{Titularité du programme}
	
	La personne physique qui l'a conçu a le droit de propriété ; le programme est protégé de par sa création, automatiquement (pas besoin de déclarer quoi que ce soit). Si un jour les droits sont transgressés, il faut prouver qu'on l'a créé.
	
	Il existe des sociétés privées où on peut déposer des logiciels, mais il faut payer un droit de dépôt. On peut aussi déposer le logiciel chez un notaire.
	
	Lorsque le programme est créé suite à un contrat avec un employeur, c'est toujours son oeuvre, mais il est présumé avoir cédé ses droits à l'employeur. Il peut y avoir des dispositions particulières dans les contrats.
	
	Est titulaire d'un programme
	
	\begin{itemize}
		\item son auteur
		\item celui à qui il a cédé les droits du programme
		\item l'employeur présumé par la loi cessionnaire des droits de ses employés, sauf disposition contraire
	\end{itemize}
	
	En Belgique, ça s'applique aux employeurs privés et au domaine publique.
	
	En tant qu'étudiant, pas de status, on a normalement les droits sauf si on signe.
	
	Dans un contrat de commande, lorsqu'une oeuvre est produire sur commande, l'auteur du programme est toujours propriétaire, sauf si cession par écrit.
	
	\section{Exercice des droits}
	
		\subsection{Droits moraux}
		
		4 grands droits moraux pour les droits d'auteur en général:
		
		\begin{itemize}
			\item droit de divulgation : on a le droit ou pas de divulguer son oeuvre
			\item droit de repentir, de retirer l'oeuvre
			\item droit de paternité : diffusion ou non en son nom
			\item droit d'opposition à la modification d'une oeuvre sans accord
		\end{itemize}
	
		Pour un programme, les droits sont moins étendus: il n'y a pas le droit de divulgation, et pas le droit d'opposition à la modification, sauf si cela porte atteinte à l'honneur et à la réputation de l'auteur.
	
		Cela s'applique même s'il y a eu cession, sauf si on l'a fait volontairement. On peut les céder partiellement, mais jamais globalement. On ne peut pas les céder à l'avance comme les droits patrimoniaux.
		
		
		\subsection{Droits économiques}
		
		3 grands droits pour l'auteur, ou bien la personne à qui il a cédé ses droits :
		
		\begin{itemize}
			\item droit de reproduction
			\item droit d'adaptation, de modification, de traduction. On ne prévient pas l'auteur si ça favorise les utilisateurs.
			\item droit de distribution et de diffusion. La première vente d'une copie d'un logiciel permet à celui qui l'a acquis de la revendre dans l'Union Européenne.
		\end{itemize}
		
		
		
		% 7 - 04 - 2011
		
		Généralement, il n'y a pas vraiment de vente, mais des distributions de copie de licence d'utilisation. On accorde par ces licences tous les droits d'exploitation.
		
		Exceptions (article 6) :
		
		\begin{itemize}
			\item Article 6 :
				\begin{enumerate}
			\item on a le droit de copier un programme pour lequel on a une licence d'utilisation, pour sauvegarder notamment ;
			\item on a le droit de le modifier seulement pour le corriger, toujours pour son utilisation normale ;
			\item Exception de contemplation : on a le droit d'essayer de découvrir les idées à la base du programme ;
				\end{enumerate}
			\item Article 7, exception de décompilation : on peut décompiler un programme pour découvrir les idées à la base, seulement pour créer un programme interopérable . Conditions de décompilation :
			
			\begin{enumerate}
				\item on doit avoir le droit d'utiliser le programme ;
				\item les conditions d'interopérabilité ne doivent pas être facilement accessibles. Ainsi, si on fournit le code source, pas le droit de décompilation ;
				\item les informations qu'on en tire ne peuvent servir que pour l'interopérabilité avec d'autres programmes. On ne peut décompiler que les parties du programme relatives à l'interopérabilité
			\end{enumerate}
			
			On ne peut pas utiliser la décompilation pour le copier et le commercialiser.
		\end{itemize}
				
				
		\subsection{Durée de la protection}
	
		Renvoi vers les conditions générales ; protection durant 70 ans après le décès du créateur.
		
		\subsection{Sanctions}
		
		La loi sur le droit d'auteur prévoit des sanctions en cas de violation (article 10). Il y a cependant des sanctions spécifiques pour les programmes (article 11).
		
		Sanctions civiles : dommages et intérêts pour le préjudice. On a aussi le droit de faire cesser l'acte illicite pour l'avenir. 
		
		
		
		