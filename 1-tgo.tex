\chapter{La naissance, la vie et les effets des contrats}

Il y a 4 différenciations à faire au niveau des contrats :


\begin{enumerate}
	\item 
	\begin{itemize}
		\item unilatéraux : contrat d'engagement d'une personne envers une autre (ex : contrat de donation, de prêt)
		\item synallagmatique : contient des obligations réciproques de 2 personnes (ex : argent contre quelque chose).
		
		On distingue ainsi
		
		
		
		\begin{itemize}
			\item le créancier, qui a un droit à l'égard d'un débiteur
			\item le débiteur, qui a une obligation envers le créditeur
		\end{itemize}
		
		Dans le cas d'un vendeur et d'un acheteur, chaque personne a les deux rôles : l'acheteur est créancier car il doit recevoir, mais est aussi débiteur car il doit payer.
		
	\end{itemize}
	
	\item 
	\begin{itemize}
		\item à titre onéreux : avec argent
		\item à titre gratuit : sans argent
	\end{itemize}
	
	\item 
	\begin{itemize}
		\item à exécution instantanée : les obligations peuvent s'effectuer en une fois (ex : vente)
		\item à exécution successive : avec échelonnement dans le temps (ex : location, remboursement)
	\end{itemize}
	
	\item 
	\begin{itemize}
		\item nommés : avec un nom au sens du droit (bail, prêt, mandat, etc), pour lesquels des règles sont prévues
		\item innommés : auxquels le droit ne donne pas de nom (leasing, sponsoring), ne sont réglementés que par la TGO 
	\end{itemize}
\end{enumerate}



	\section{Formation des contrats}
	
	Le code civil contient de nombreuses règles à propos des contrats, qu'elles s'appliquent à tous les contrats (ex : ce qui se passe lors de violation d'obligations) ou bien à des contrats spécifiques (ex : dans une vente, qui livre, et où on livre). Leur but est 
	
	\begin{itemize}
		\item déterminer les critères pour qu'un accord soit qualifié de contrat
		\item prévoir un régime supplétif pour la vie du contrat ; cela évite de régler tous les points d'un contrat, de proposer un modèle qui s'appliqueront si rien de contraire n'est vu
	\end{itemize}	
	
	Le régime légal ne s'applique qu'en cas de silence des parties, ce sont des \underline{règles supplétives} (elles suppléent au silence des parties).
	
	
		\subsection{Grands principes}	
	
	Deux grands principes sont respectés :
	
			\subsubsection{La liberté contractuelle}
			
			Chacun est libre de 
		
			\begin{enumerate}
				\item contracter ou pas
				\item de contracter avec qui on veut
				\item  de contracter sur ce qu'on veut
			\end{enumerate}
		
			Il y a cependant des exceptions :
		
			\begin{enumerate}
				\item on est tenu de faire un contrat avec les assurances-voitures (RC, pas omnium)
				\item on n'est pas tenu de contracter avec qui on veut (ex du locataire d'un terrain : si le terrain est à vendre, on est obligé de le vendre au locataire en priorité)
				\item on ne peut pas vendre ce qu'on veut (ex : des infractions pénales, des organes, \dots )
			\end{enumerate}
		
			\subsubsection{Le consensualisme}
			
			Liberté au niveau de la forme ; il n'y a pas de forme imposée pour un contrat. Il n'est pas nécessairement écrit (peut être oral, avec témoins), mais c'est conseillé car cela constitue une \underline{preuve}.
		
			Exceptions : un contrat de mariage, un dérogation de vote et un contrat d'hypothèque doivent se faire par écrit.
			
		\subsection{Avant le contrat : la négociation}
		
		On est libre de négocier, mais avec certaines réserves :
		
		\begin{enumerate}	
			\item on doit négocier de bonne foi (principe moral et légal). En informatique, avec un fournisseur et un utilisateur, on doit
			
			\begin{itemize}
				\item s'informer précisément sur les besoins de l'utilisateur
				\item l'informer sur le contenu de l'offre
				\item informer sur l'utilité du programme
			\end{itemize}	

		
			L'acheter a pour obligation de définir au mieux son problème : il doit exprimer ses besoins et les objectifs qu'il veut atteindre, notamment en indiquant la nature et le volume du travail à automatiser. Il doit aussi faire lui-même le calcul de rentabilité de l'investissement.
		
			Le fournisseur a trois obligations :
		
			\begin{itemize}
				\item il doit s'informer des besoins de l'utilisateur, et devra demander des informations complémentaires si celles qu'il fournit semblent contradictoires ou insuffisantes : il doit aider l'utilisateur a exprimer les données de son problème ;
				\item il doit informer l'utilisateur du contenu de la proposition qu'il fait (caractéristiques du matériel, conditions d'utilisateurs, etc) ;
				\item il doit conseiller l'utilisateur : il doit apprécier l'opportunité de la démarche d'informatisation, et ne peut se contenter de proposer n'importe quel matériel. Il doit propoer n matériel adapté aux besoins.
			\end{itemize}
		
			\item la liberté d'arrêter de négocier, sans être de mauvaise foi. 
			
			Dans certains cas, des sanctions sont possibles (dommages et intérêts) si les négociations sont fortement avancées, et si des frais importants ont été engagés.
			
			La bonne foi dans la négociation n'implique pas qu'on doit parvenir un accord.
		\end{enumerate}		
		
		S'il y a de la mauvaise foi ou une rupture abusive, des dommages et intérêts devront être payés par le fautif selon si le contrat a été conclu ou non.
		
		\begin{itemize}
			\item Si le contrat a été conclu, la partie trompée peut
			
			\begin{itemize}
				\item demander l'annulation et des dommages et intérêts pour le dommage subit
				\item maintenir le contrat, et demander des dommages et intérêts (ex : pour du matériel non adapté, engager des personnes en plus pour le gérer)
			\end{itemize}
			
			\item Si le contrat n'a pas été conclu, des dommages et intérêts peuvent être demandés pour les frais exposés, la perte de temps, des occasions de traiter avec d'autres personnes perdues, etc. 
			
			On ne peut pas demander à titre de dommage les bénéfices que le contrat aurait engrangé, car il n'a pas été conclu.
		\end{itemize}
		
		
		\subsection{Formation du contrat}
				
		Il faut une offre (volonté d'être engagé) d'une part et une acceptation d'autre part pour que le contrat soit formé.
		
			\subsubsection{L'offre}
		
			Pour être valable, l'offre doit être
		
			\begin{itemize}
				\item ferme : une offre sous-réserve n'est pas une offre, il y a un engagement.
				\item précise : comporte des précisions sur les éléments essentiels du contrat. A l'opposé, l'offre ne doit pas porter sur tout (ex : vente de voiture : prix et la chose). On peut rajouter des points (ex: la livraison). Il y a trois types d'éléments :
			
				\begin{itemize}
					\item essentiels
					\item accessoires
					\item essentialisés : points qui sont importants pour un parti ; accessoires rendus essentiels.
				\end{itemize}
			\end{itemize}
		
		
			\subsubsection{L'acceptation}
		
			L'acceptation est l'accord sans condition. Changer un élément pour accepter n'est pas une acceptation mais une nouvelle offre.
		
			On peut accepter une offre explicitement (par écrit, verbalement) ou tacitement, par le silence. Le problème est de savoir si ce dernier correspond à un consentement ou non. 
		
			Juridiquement, on considère que ce n'est pas un consentement sauf si le silence est circonstancé (entouré de circonstances qui font croire que la personne accepte) (ex : relations entre commerçants). En cas de refus, il doit y avoir une protestation.
		
			C'est pour cela qu'il est important de savoir si on a affaire à un particulier ou à un commerçant : si un particulier répond par le silence, ce n'est pas une acceptation, donc pas une preuve du contrat. Si le commerçant ne dit rien, le contrat est implicitement accepté.
		
		
			L'offre et l'acceptation peuvent être faits par écrit (papier ou mail), mais peut ne pas être formalisé par écrit (consensualisation). Le fait de ne rien dire dans certains cas fait présumer l'acceptation.
		
			L'offre est un engagement unilatéral : si on fait une offre, on ne peut plus la retirer ni la modifier, même s'il n'y a pas eu acceptation (sauf si l'offre n'est pas arrivée à l'autre parti). Mais l'offre a une durée limitée (que l'on indique ou après un délai raisonnable (selon un juge)), elle ne peut être réutilisée des années après).
	
			Si on émet une offre de 30 jours, mais qu'on la retire après 15 jours, et qu'on veut en profiter au 20ème jours, par principe, l'offre doit être maintenue. Dans les faits, on ne peut obliger l'offre, mais on peut réclamer une sanction (dommages et intérêts).
		
			\subsubsection{La réalité des négociations}		
				
			Dans une vrai négociation, beaucoup plus compliqué qu'une offre et une acceptation. Si la négociation est bien avancée, il y a un accord implicite (par exemple s'il ne reste que des éléments accessoires). Si on veut éviter qu'on considère qu'un contrat est passé, il faut dire clairement qu'il n'y en a pas. Au début d'une négociation, on peut envoyer une lettre d'intention (gentlemen's agreement) qui ne fait que déclarer l'intention avec quelques éléments, mais qui stipule que ce n'est pas un contrat (juste une déclaration en entrée de négociation).
		
				Dans les négociations, il y a des engagements définitifs, des contrats conclus pour encadrer la négociation. Il y a notamment 
		
			\begin{itemize}
				\item des accords de confidentialité, signés par les deux partis, par exemple s'il faut des informations sensibles pour établir le prix de l'offre. Cela marche aussi dans l'autre sens.
				\item des accords d'exclusivité de négociation du contrat
			\end{itemize}
		
			\subsubsection{Rupture des négociations}
		
			Les parties sont en droit de rompre les négociations à tout moments, sauf s'il y a des abus. [ p9]
		
		\subsection{Validité d'un contrat}
		
		Il doit, pour être valable, répondre à certaines conditions :
		
		\begin{itemize}
			\item il y a un consentement des deux côtés, il doit être parfait et non faussé. Un consentement peut être vicés (vices de consentement), l'accord d'un parti est altéré. 
			
			Cas où un accord est vicé :
			
			\begin{enumerate}
				\item Cas d'une erreur valable :
			
				\begin{enumerate}
					\item lors de l'approbation, lorsque l'erreur porte sur la substance ou la qualité de la chose, et pas sur la valeur (par exemple une voiture pas assez puissante pour ce qu'on veut en faire. Ce n'est pas une erreur si on paie plus cher que chez quelqu'un d'autre).
					\item l'erreur doit être déterminante dans l'accord (si on avait su, on n'aurait pas conclu)
					\item l'erreur doit être commune
					\item l'erreur doit être excusable
				\end{enumerate}
			
				\item Dol : manoeuvres adoptées par un parti pour tromper l'autre. 
				
				Cela peut être des mensonges (le fait de vanter un produit n'est pas du dol) positifs (mentir pour que la vente soit conclue) ou négatifs (rétention d'informations qui feraient capoter la vente) qui ont déterminés la vente. 
				
				Il n'y a pas l'exigence que c'est excusable. Il y a dol s'il émane d'un parti qui conclue le contrat, et non d'un tiers. 
				
				De plus, il doit avoir déterminé la conclusion du contrat, sinon c'est un incident. Dans le cas d'un incident, on ne peut pas demander l'annulation du contrat.
								
				\item Violence, on n'est pas libre de marquer ou non son accord. La violence peut être utilisée par le cocontractant, ou par un tiers. Cela conduit à l'annulation du contrat si 
				\begin{itemize}
					\item le mal est sérieux, porte atteinte à la personne (physique ou moral) ou à l'entourage ;
					\item détermine le consentement ;
					\item est illégitime. 
				\end{itemize}
				
				\item Lésion, disproportion des prestations, seulement dans le cas où la lésion est qualifiée : on utilise la faiblesse de quelqu'un (ignorance caractérisée).
			\end{enumerate} 
			
			\item l'accord doit émaner de personnes capables (en droit) de faire des contrats (ex : un mineur ne peut acheter un immeuble). 
			
			Cela vaut aussi pour les sociétés (ex : un employé ne peut acheter un immeuble pour son entreprise, le patron ou les administrateurs peuvent).
			
			\item le contrat doit avoir un objet, il y a une condition relative à l'objet. L'objet doit être
			
			\begin{itemize}
				\item (techniquement) possible
				\item déterminé ou déterminable (prix ou obligation qui peut être fixé à l'instant présent). Exemple de prix déterminable : essence (déterminé par le gouvernement), bourse
			\end{itemize}
			
			\item le contrat doit avoir une cause, il y a une raison pour laquelle on a fait un contrat (ex : cas de la vente d'un bâtiment avec annexe. Si on annule la vente du bâtiment, l'annexe n'est pas non plus vendue).
			
			La cause est déterminée subjectivement, et les parties doivent la connaître.
			
			\item il doit être conforme à l'ordre publique (pas contre les lois) et aux bonnes moeurs. L'ordre publique comprend les règles fondamentales pour la vie en société.
			
			Deux types de règles : règles supplétives (modifications en plus par rapport au TGO) et règles à l'ordre publique (règles pénales, accords illicites et règles de la concurrence, règles de baille).
		\end{itemize}
		
		S'il y a un problème, la sanction est l'annulation/la nullité du contrat (au niveau de la forme), dans un délai de 10 ans.
		On fait comme si le contrat n'a jamais existé s'il n'a pas encore été exécuté. 
		
		S'il a commencé à être exécuté, il y a une restitution de ce qui a été donné/reçu. 
		
		Deux exceptions : 
		
		\begin{itemize}
			\item celui qui est le plus coupable pourra ne pas recevoir tout ce qu'il doit recevoir.
			\item refus de la nullité si l'un des parties en tire un avantage, en abusant de la sanction.
		\end{itemize}
		
		La personne qui a été lésée peut demander la nullité, mais elle peut l'accepter : c'est une \underline{confirmation de nullité}, le contrat n'est pas anéanti. N'est pas confirmable si ce n'est pas contraire à l'ordre publique ou venant de personnes non capables.
		
	\section{Les effets du contrat}
	
	Deux grands principes :
	
	\begin{enumerate}
		\item principe de convention-loi : le contrat est quelque chose qui a la même force, la même obligation qu'une loi.
		\item principe de l'exécution de bonne foi
	\end{enumerate}
	
		\subsection{Principe de convention-loi}
		Une fois qu'on a accepté, on ne peut pas résigner un contrat, sauf dans 3 cas :
	
		\begin{itemize}
			\item dans un commun accord
			\item lorsque la loi le prévoit (ex : le mandat, que l'on peut arrêter quand on veut, mais avec frais) 
			\item lorsque le contrat le prévoit
		\end{itemize}
		
		Des évènements peuvent rendre impossible l'exécution d'un contrat (cas de force majeure). Cependant si c'est toujours possible, même si c'est plus cher à exécuter, on peut forcer l'exécution. La Belgique n'accepte pas l'imprévision. 
		
		On peut prévoir une \underline{clause hardship}, qui stipule que si des évènements sérieux imprévisibles surviennent et déséquilibrent le contrat, par avance, on est d'accord pour renégocier le contrat. Elle permet d'annuler la théorie de l'imprévision.
	
		On ne peut donc modifier le contrat (sauf si clause de hardship), pas plus qu'un juge.
	
		% Cours 24-2-2011
	
		\subsection{Exécution de bonne foi}
		
		Les contrats doivent être exécutés de bonne foi. 
		
		On distingue 3 composantes :
		
		\begin{itemize}
			\item fonction interprétative : on doit interpréter un contrat de bonne foi (ex : ne pas livrer un cheval mort quand on vend un cheval).
			\item fonction complétive : dans certains cas, on est obligé de faire ce qu'il est écrit, mais aussi de faire des choses pas spécialement spécifiées. En informatique particulièrement,
			
			
			\begin{itemize}
				\item obligation d'information : on doit conseiller et informer son client
				\item obligation de sécurité
			\end{itemize}
			
			Cela fonctionne du côté du vendeur et du client (qui doit informer de ses besoins par exemple).
			
			\item fonction modératrice, qui évite l'abus de droit (invoquer un droit pour retirer un avantage disproportionné par rapport à l'inconvénient causé à l'autre) (ex : louer un kot pour 5 ans, s'en aller après 3 mois, et réclamer les 5 ans de loyer)
		\end{itemize}
		
		
	\section{Inexécution du contrat}
	
	Contrat pas exécuté ou mal exécuté, d'un parti ou de l'autre. On ne peut utiliser la force, mais on peut forcer l'exécution  ou le payement de la prestation en passant par un juge ou un arbitre (justice privée).
	
	L'arbitrage est une manière de régler les conflits, mais de manière privée.
	
	
	\begin{center}
	\begin{tabular}{c|c}
		\textbf{Juge} & \textbf{Arbitrage} \\ 
		\hline\hline nommé par l'Etat & nommé par les partis \\ 
		\hline "lent" & rapide \\ 
		\hline public & privé et confidentiel (1) \\ 
		\hline généraliste & spécialiste \\ 
		\hline gratuit & payant \\ 
		\hline possibilité de faire appel & pas d'appel \\
		\hline 
	\end{tabular} 
	\end{center}
	
	(1) : pratique quand cela concerne des données sensibles et les droits d'auteur.
	
	Dans les deux cas, on rend un jugement ; un juge rend un jugement, un arbitre rend une sentence. Ils sont tous les deux définitifs, mais le jugement a la force exécutoire (rendu par l'autorité, l'Etat ; on peut recourir à la police, à des huissiers, etc), pas la sentence (mais on peut quand même l'obtenir avec une procédure rapide avec un juge).
	
	On peut prévoir une clause qui spécifie qu'on n'ira pas devant le juge en cas de litige (en dehors du contrat, régime propre).
	
	Etapes pour forcer l'exécution d'un contrat : 
	
	\begin{itemize}
		\item mise en demeure du débiteur
		\item aller devant le juge (ou l'arbitre) pour avoir un titre exécutoire
	\end{itemize}
	
		\subsection{La mise en demeure}
		
		Cela signifie rappeler une ultime fois officiellement l'exécution de l'obligation (le délai de payement/livraison ne suffit pas). Généralement, 2 formules :
		
		\begin{itemize}
			\item par un huissier de justice
			\item par une lettre recommandée
		\end{itemize}
		
		On peut mettre en demeure un commerçant par un coup de téléphone ou une simple lettre, mais on doit en avoir une preuve.
		
		Deux exceptions qui ne nécessitent pas la mise en demeure :
		
		\begin{itemize}
			\item si les parties l'ont prévu dans le contrat (la mise en demeure est une règle supplétive)
			\item c'est inutile (ex : engagement de confidentialité, où on ne saurait plus respecter l'engagement)
		\end{itemize}
		
		\subsection{Titre exécutoire}
		
		C'est un acte authentique qui permet au créancier de forcer l'exécution d'un contrat (sinon il devrait aller devant les tribunaux).
		
		On doit démonter qu'on a raison et que l'inexécution de l'autre partie est de sa faute. On veut obtenir \underline{l'exécution forcée} 
		
		\begin{enumerate}
			\item en nature
			\begin{itemize}
				\item directe
				\item indirecte
			\end{itemize}
			\item par équivalent
			\item si contrat synallagmatique,
			\begin{itemize}
				\item exception d'exécution
				\item résolution
			\end{itemize}
		\end{enumerate}
		
		Dans les cas 1 et 2, on doit démontrer la responsabilité du débiteur, variable selon une obligation de moyen ou de résultat.
		
		
			\subsubsection{En nature}
			
			Obtenir en jugement forçant le débiteur à exécuter le contrat. On peut y ajouter des astreintes.
			
			Une astreinte est un procédé qui fait payer celui qui est condamné et qui ne se conforte pas au jugement (ex : par jour de retard, on doit payer un montant). Cet argent sera dû au delà de ce qu'on doit faire. Elle ne peut viser que ce qui porte sur une prestation matérielle, pas sur l'obligation à payer une facture.
			
			L'exécution peut être directe ou indirecte : on demande au juge que l'exécution soit faite, et que si l'autre partie ne s'exécute pas, on puisse faire appel à quelqu'un d'autre.
			
			L'exécution en nature peut ne pas être possible :
			
			\begin{itemize}
				\item matériellement impossible (ex : site web pour élection disponible après le jour de l'élection)
				\item immoral (on ne peut forcer quelqu'un de son plein gré)
			\end{itemize}
		
			Dans ces cas, demande d'exécution en nature indirect ou par équivalent.
			
			\subsubsection{Par équivalent}
			
			Demande d'argent, autant que ce qu'on aurait dû avoir.
			
			Il faut pour cela montrer que l'autre partie est fautive, que le contrat n'est pas exécuté \underline{à cause} d'elle (pas seulement non exécuté) ; pour prouver la responsabilité, il faut montrer que
			
			 \begin{itemize}
				\item l'inexécution doit lui être imputable.
				\item le préjudice
			\end{itemize}
			
			
			\paragraph{Faute imputable}
			
			Il faut d'abord savoir si la dette/l'obligation est dite
			
			\begin{itemize}
				\item de moyen : on doit tout mettre en oeuvre pour atteindre le résultat, mais s'il n'est pas atteint on n'est pas totalement responsable (ex : médecin/avocat a des obligations de moyen)
				\item de résultat : quelqu'un est tenu de fournir un résultat (ex : tenu de payer, de fournir un site web dans un tel délai ; un médecin peut avoir des obligations de résultat, par exemple ne pas oublier des instruments de quelqu'un). Elle est sûr et certaine de réussir, ne dépend que d'un parti.
			\end{itemize}
			
			On détermine le type d'obligation sur base des critère de l'aléa, à priori. 
			
			Lorsque c'est une obligation de résultat, on est automatiquement présumé en faute. Si c'est une obligation de moyen, il faut démontrer en plus que l'autre partie est en faute, en ayant recours à des experts.
			
			Une fois le problème posé, on est présumé en faute, sauf s'il y a une cause étrangère libératoire, autrement dit un \underline{évènement} extérieur qui \underline{rend impossible l'exécution du contrat} \underline{sans la faute du débiteur} :
			
			\begin{itemize}
				\item force majeure ou le cas fortuit (tremblement de terre)
				\item le fait du créancier (ex : on interdit l'accès à des ordinateurs pour y installer un logiciel)
				\item le fait du prince, une décision d'État qui empêche l'exécution (ex : embargo).
				\item fait d'un tiers
			\end{itemize}
			
			Ce qui se produit doit rendre impossible le contrat (sinon hardship).
			
			On peut aussi prouver que ça ne peut pas être notre faute (ex : freins qui lâchent alors qu'il y a eu entretien)
			
			\paragraph{Prouver le dommage}
			
			Il faut montrer qu'il y a eu perte, qu'il y a eu préjudice, ce qui est difficile à estimer. Cela comporte les frais exposés et le bénéfice potentiel (qui est compté car le contrat est en cours d'exécution).
			
			Le créancier ne peut faire état que d'un dommage direct et immédiat de l'inexécution (il doit prouver le lien), et sauf dol, seuls les dommages prévisibles lors de la conclusion du contrat peuvent être réparés (il suffit que les parties aient prévu le principe même du dommage, non son montant).
			
			On peut prévoir des clauses pour éviter les procès et les frais :
			
			\begin{enumerate}
				\item \underline{Clauses pénales} : fixent le montant du dommage qu'on subira que l'on risque de subir si l'exécution n'est pas respectée (ex : acompte). Cela permet de gagner du temps et de l'argent par après.
			
				La différence avec l'astreinte est que cette dernière est donnée par le juge, tandis que la clause pénale est définie par les parties.
			
				La clause pénale est valable sauf si elle est excessive, où un parti a plus intérêt à ce que le contrat ne soit pas exécuté. Seul un juge peut la réduire (seul cas où il peut intervenir sur un contrat).
			
				Dans certains cas, une clause peut lever les responsabilités : ce sont des clauses limitatives.	

				% 3 - 3 - 2011			
			
				\item \underline{clauses de non responsabilité ou de limite de responsabilité} : limitent ou enlèvent la responsabilité lors d'une inexécution (ex : responsable d'un certain type de problème et pas d'autre, ou bien responsable pendant un an, ou bien responsable jusqu'à un certain montant (celui de la facture par ex)).
						
				Ces clauses ne sont pas valables lorsque
				\begin{itemize}
					\item elles sont excessives, au point où on ne s'engage à plus rien ; si elles enlèvent tout objet du contrat
					\item il y a dol, lorsque la personne a de manière intentionnelle inexécuté le contrat, par malhonnêteté ; le client pourra réclamer totalement son dommage, pas ce qui est limité
					\item la loi l'interdit, par exemple la garantie de 2 ans minimum qu'on ne peut diminuer (lors de vente de professionnel à consommateur, entre consommateur ce n'est pas le cas)
					
					\item elles libèrent le fournisseur d'une inexécution d'une obligation ou d'une prestation du contrat. Par exemple, s'il y a une clause "l'utilisateur reconnait avoir été parfaitement renseigné et conseillé", cette clause libère le fournisseur de son devoir de renseignement et de celui de conseil, ce qui le libère également de son obligation de livrer un matériel adapté, ce qui est la prestation principale du contrat.
				\end{itemize}
				
				
			
				\item \underline{Clause de force majeure} , pour un évènement imprévu et rendant impossible l'exécution ($\neq$ imprévision, qui rend plus difficile l'exécution du contrat, mais pas son impossibilité). Elles servent à s'arranger pour que certains évènements soient considérés comme des cas de force majeure, alors que techniquement ce n'en sont pas (ex : grève).
				
				Dans le cas de grève, distinction entre grève nationale et interne. S'il y a grève interne avec raison, c'est la faute du patron.
			\end{enumerate}			
			
			
			\subsubsection{Cas d'un contrat synallagmatique}
			
			Deux méthodes supplémentaires dans ce cas uniquement pour le créancier.
			
				\paragraph{L'exception d'inexécution}
			
				 L'exception du contrat non exécuté (correctement) est un moyen passif : lors d'une succession d'exécutions, on peut ne rien faire tant que l'autre partie n'a pas fait ce qu'elle devait faire avant (suspension d'exécution).
				
				Conditions pour que ça marche :
				
				\begin{itemize}
					\item il faut que ça soit un contrat à exécutions successives (et non instantanées) (ex : acompte)
					\item l'inexécution de l'autre soit réelle et effective/consommée ; on ne peut pas refuser de s'exécuter si on présume à l'avance du problème chez l'autre.
					\item cela doit être fait de bonne foi, et proportionnel à l'inexécution causée
				\end{itemize}
				
				Il n'est pas nécessaire de passer par un juge, c'est un moyen de pression.
				
				\paragraph{L'action en résolution}
				
				Moyen offensif : action introduite devant le juge, et demandant la résolution du contrat.
				
				Pour que ça soit valable, il faut que ça soit grave. Il y a alors restitution réciproque des prestations. Si le contrat est à exécution successive, la résolution n'opère que pour l'avenir ; les prestations passées restent acquises.
				
				\underline{Tuyau} : la résolution est la fin d'un contrat bien formé, mais qui s'est mal terminé pendant l'exécution. L'annulation se produit lorsque le contrat est  mal formé dès le début.
				
				Cependant, il faut des années de procédures : s'il y a urgence, on autorise les partis à résoudre unilatéralement le contrat, sans passer par un juge. On peut également demander des dommages et intérêts à la partie incriminée.
			
			
			
				\paragraph{Clause résolutoire}
			
			Pour éviter d'aller devant le juge et que ça ne traîne, on peut prévoir une \underline{clause résolutoire expresse}, qui précise quels sont les cas considérés comme graves (ex : très léger défaut de payement). Il est quand même nécessaire de mettre en demeure (sauf si clause de non mise en demeure).

			
	\section{Champ d'application du contrat}
	
		\subsection{Le principe de la relativité des conventions}
		
		Un contrat ne concerne et ne lie que les personnes qui l'ont conclu : seules ces personnes ont des droits et des obligations.
		
		Des personnes cependant peuvent être considérées comme partis, alors qu'elles ne sont pas présentes lors de la signature du contrat. 
		
		Par exemple : - si quelqu'un qui a signé un contrat décède, c'est la famille qui doit s'en charger - un mandataire, qui signe au nom de quelqu'un d'autre.
		
		\pagebreak
		Exceptions :
		\begin{itemize}
			\item On ne peut jamais assigner des obligations ou une dette à un débiteur à un contrat dont il ne faut pas partie. 
			
			En revanche, des personnes tierces peuvent être titulaires d'une créance ; on donner des droits à des personnes tierces, jamais des obligations (ex: assurance, qui peut assurer contre le vol notre matériel, ainsi que le matériel d'autrui entreposé chez soi) : c'est la stipulation pour autrui.
			
			\item hypothèse de l'action directe : cas où le client fait appel à un prestataire, qui fait lui-même appel à un sous-traitant. Selon la loi, le sous-traitant a un droit d'action directe sur l'acheteur, au cas où le sous-traitant ne serait pas payé (l'acheteur doit ne pas avoir payé le fournisseur).
			
			Si le sous-traitant n'exécute pas bien son contrat et que le fournisseur fait faillite, le client ne peut rien réclamer au sous-traitant (contre la logique, mais la loi ne le prévoit pas), car il ne peut intervenir dans le contrat fournisseur-sous-traitant. De plus, le fournisseur ne peut pas refuser de payer le sous-traiteur si le client ne paye pas.
		\end{itemize}
		
		
		\subsection{Principe de l'opposabilité}
		
		
		Le contrat lie deux parties, mais existe et peut être opposé à quelqu'un, montré à tous (ex : si on loue un appartement et si le bâtiment est vendu, il y a opposabilité avec le nouvel acquéreur).
		
	
	\section{Interprétation des contrats}
	
	L'important est d'aller au delà des mots du contrat, et si ce n'est assez précis, interpréter.
	
	Principes directeurs :
	\begin{itemize}
		\item Le principe de base est de trouver la volonté réelle et commune des parties.
		\item Le juge est limité par le principe de la convention-loi, qui stipule qu'il doit se conformer à ce qui y se trouve dans le contrat.
		\item le juge doit respecter la force probante des écrits et la foi qui leur est due
		\item le législateur a prévu des conseils d'interprétation.
	\end{itemize}