\chapter{Commerce électronique}

	\section{La publicité}
	
	Publicité : toute forme de communication destiné à promouvoir directement ou indirectement des biens, des services ou l'mage d'une entreprise.
	
	Deux grands types de loi : générales et spécifiques. Deux lois importantes :
	
	\begin{itemize}
		\item Loi sur les pratiques du marché (6 avril 2010), ne protège que les consommateurs ;
		\item Service de la société d'information (11 mars 2003) : exclut les infos sur les entreprises (ex : adresse e-mail), protège les consommateurs et les fournisseurs.
	\end{itemize}
	
	La difficulté est de faire la séparation site d'information/publicité. Ne sont pas considéré comme de la publicité : avoir un site informatif, ou donner une information. La différence est importante, car les publicités sont régies par des lois spécifiques.
	
	4 grands principes :
	
	\begin{enumerate}
		\item principe d'identification (11 mars 2003) : on doit savoir clairement que la communication est une publicité (publireportage compris) ;
		\item principe de transparence et de loyauté (2003) : la publicité va permettre l'identification de la personne qui l'a émise, on doit savoir qui est derrière ;
		\item la publicité trompeuse ou mensongère est interdite (2010) ;
		\item la publicité comparative est autorisée, mais est réglementée. La comparaison doit être objective et sans dénigrer le concurrent.
	\end{enumerate}
	
		\subsection{Spamming}
			
		Envoi massif, parfois répété, de courrier électronique non sollicité, souvent à caractère commercial, sans spécialement avoir de contact avec l'expéditeur.
		
		Il est soumis aux règles plus haut et aux législations à la vie privée. Deux systèmes sont à envisager :
		
		\begin{itemize}
			\item opt-out : l'envoi de courrier non sollicité est permis, mais à partir du moment où le destinataire veut  arrêter, il faut le faire ;
			\item opt-in (2003) : on ne peut envoyer du courrier que si on a eu le consentement du destinataire (dans le système belge).  Caractéristiques du consentement :
			
			
			\begin{itemize}
				\item on doit être libre ;
				\item une personne qui refuse de le donner ne doit pas être discriminée ;
				\item le consentement est spécifique, ne concerne que le prestataire (on ne peut pas consentir à quelqu'un et recevoir les mails d'une autre personne) ;
				\item on doit être informé (on doit savoir que c'est à des fins de marketing).
			\end{itemize}
			
			Il n'y a pas de forme imposée (par écrit, bon de commande, coup de téléphone, etc). On pourrait envoyer un mail pour demander l'autorisation, mais c'est de l'opt-out (qui est interdit).
			
			La loi de 2003 donne la possibilité de s'opposer, à tout moment on peut montrer sa volonté de ne plus en recevoir.
		\end{itemize}
		
		Marketing viral : marketing qui se fait par les pair, permet de contourner les lois (par exemple envoyant un mail à partir de l'adresse d'une personne qui aurait joué à un jeu, à partir du mail de l'entreprise serait bon ; une entreprise à besoin du consentement, pas une personne).
	
		\subsection{Processus d'acceptation des contrats}
	
		Les contrats à distance sont régis par les droits communs, mais où les parties ne sont pas mise en présence l'une de l'autre. Des différentes législations peuvent se superposer.
		
		Avant de conclure le contrer, des informations à communiquer :
		
		\begin{itemize}
			\item informations générales (2003) : le prestataire assure un accès facile, direct et permanent à toute une série d'informations, qu'ils soient professionnels ou consommateurs (ex : nom, adresse, TVA du prestataire). Si un prix est mentionné, il faut qu'il soit affiché clairement et en précisant si les taxes et frais sont compris.
			
			La loi de 2003 n'oblige pas la communication d'un prix, sauf 2010 (?).
			
			\item informations sur le processus contractuel : avant que le destinataire ne passe commande, il faut lui expliquer comment cela va marcher : langues proposées pour la conclusion, les différentes techniques à suivre pour conclure le contrat. Il faut aussi montrer les techniques pour éviter une erreur (avec des messages récapitulatifs).			
			
			\item infos spécifiques face à un consommateur : coordonnées du vendeur, caractéristiques essentielles du produit ou du service, les modalités de payement, de livraison et d'exécution du contrat, s'il y a ou non un droit de rétractation.
			
			On définit les conditions générales de vente, qui servent aux vendeurs à harmoniser les techniques de vente : tous les contrats auront les mêmes règles. L'entreprise qui les oppose à un consommateur doit prouver que ça entre dans le champ contractuel, c'est-à-dire que la personne a pu en prendre connaissance (sinon elles ne sont pas opposables). Ces clauses doivent être communiquées pour qu'elles soient conservables et recopiables.
		\end{itemize}
	
		\subsection{Conclusion du contrat}
		
		Un contrat démarre comme en droit commun, lorsqu'un offrant a connaissance de l'acceptation. Lorsqu'on est face à des consommateurs, une fois la commande passée, il y a encore des informations à faire parvenir. On doit confirmer
		
		\begin{itemize}
			\item les données de l'offre ;
			\item si le consommateur a un droit de rétractation ou non (il y a des exigences de forme : dans cadre distinct du texte, en 1ère page et en gras).
		\end{itemize}
		
		Doit se faire par écrit ou sur un support durable.
		
		Cette confirmation se fait après la livraison des produits, alors que pour des services cela se fait avant qu'ils soient effectués.
			\subsubsection{Droit de rétractation}
			
			Légalement ne s'applique que pour un contrat à distance, qui est de 14 jours calendrier. Le consommateur ne doit pas nécessairement donner des motifs ni payer un supplément, seul les frais de renvoi sont à sa charge.
			
			Ces textes sont impératifs, on ne peut y déroger.
			
			Exception :
			
			\begin{itemize}
				\item quand un contrat a débuté dans son service, après le délai de rétractation, mais avec l'accord du consommateur (ex : exécution plus tôt) ;
				\item si la commande est spécifique (ex : un t-shirt avec une photo), idem pour la nourriture (périssable), les magazines, des périodiques ou des logiciels informatiques ;
.				\item certains services planifiés (ex : chambre d'hôtel loue sur une certaine période)
			\end{itemize}
			
			Sanctions lourdes : si le droit de rétractation n'est pas précisé ni confirmé, le droit de rétractation passe de 14 jours à 3 mois. Si la forme n'est pas respectée, le consommateur peut consommer le bien sans le payer ; c'est assimilé à de la vente forcée.
			
			\subsubsection{Modes de payement}
			
			Carte de crédit, Paypal, virement, à la livraison, etc.
			
			Loi de 2009, qui s'applique aux services de payement. Dans un service de payement, un consommateur qui agit de manière non professionnelle.
			
			Obligations de l'utilisateur :
			
			\begin{itemize}
				\item doit utiliser le moyen de payement conformément à l'utilisation
				\item doit prévenir en cas de vol ou de dommage de l'instrument de payement
				\item doit préserver la sécurité de l'instrument
			\end{itemize}
			\pagebreak
			Obligations du prestataire :
			
			\begin{itemize}
				\item assurer la confidentialité ;
				\item donner les moyens d'être averti d'un problème avec la carte et de la bloquer.
			\end{itemize}
			
			La loi prévoit un partage de responsabilité entre l'utilisateur et le prestataire, qui s'articule sur le moment de la notification ; tant que l'utilisateur n'a pas prévenu la banque, c'est lui qui est responsable (avec un plafond de 150 euros), qui peut tomber s'il y a négligence grave ou s'il a agit intentionnellement. Dès que c'est notifié, c'est le prestataire qui supporte le dommage.
			
			Si l'utilisateur n'a commit aucune faute, il ne supportera pas de charge si
			
			\begin{itemize}
				\item pas d'identification électronique et ni de présentation physique de l'instrument (on ne montre pas la carte ou on ne signe pas électroniquement)
				\item utilisation par un tiers alors qu'on possède toujours l'instrument
			\end{itemize}
			
			
			
		\subsection{Protection des données personnelles}
			
		Assurée par une loi de 1992, qui s'applique aux données à caractère personnelle, c'est-à-dire toute information qui rend une personne physique identifiée ou identifiable (numéro de compte, adresse, nom, etc).
		
		On ne peut pas traiter des données personnelles, sauf des exceptions (quand la personne concernée a consenti, ou si les données sont nécessaires pour le contrat). Si on les traite, on doit le faire loyalement, à des fins légitimes. On doit informer la personne et lui donner la possibilité d'y avoir accès ou de les modifier, ou de s'opposer au traitement. Les données concernant la religion, les convictions politiques, etc ne peuvent être conservées.