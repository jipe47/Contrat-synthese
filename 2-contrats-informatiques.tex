\chapter{Les contrats informatiques}
	
Généralement, les contrats sont imbriqués les uns dans les autres.

Si des contrats dépendent de l'exécution d'autres contrats, on peut mettre des \underline{conditions suspensives} : un contrat sera exécuté si un autre est correctement exécuté  Cela concerne des événements futurs et incertains ; les effets du contrat sont suspendus jusqu'à ce qu'un évènement se produise (ex : lors de l'achat d'une maison, accord avec le vendeur, mais crédit hypothécaire non obtenu).

On distingue la condition suspensive de la condition résolutoire, où le contrat est résolu à cause d'un évènement extérieur, alors que pour une condition suspensive le contrat n'a pas commencé. (clause résolutoire relative à une inexécution).

% Notes 10 - 3 - 2011

(tuyau) Les conditions suspensives lient donc des contrats différents, suspendent les contrats jusqu'à l'exécution d'un évènement extérieur réalisable (ex : demander un prêt, acheter un immeuble et installer un parc informatique).


	\section{Contrat hardware}
	
	La notion de possession est différente de celle de propriété ; posséder quelque chose ne signifie pas qu'on est propriétaire (il est juste présumé propriétaire).
	
	
		\subsection{Obligations du  vendeur}
	
			\subsubsection{Transfert de la propriété}
			
			Par le fait qu'un vendeur et un acheteur sont d'accord, la propriété est transférée de manière automatique (il n'est donc pas nécessaire que ça soit écrit ou qu'il y ait eu transfert pour être propriétaire). La propriété est transférée même sans livraison ni payement, juste avec accord. 
		
		
			Exceptions :
		
			\begin{itemize}
				\item tout ce qui concerne des choses futures, lorsque l'objet n'existe pas encore. Le transfert de propriété sera effectif quand il existera ;
				\item lorsque la chose est de genre, c'est-à-dire lorsque la chose n'est pas individualisée (par exemple un élément d'une cargaison, parmi plusieurs). Lorsqu'elle le sera, le transfert de propriété sera effectif ;
				\item avec une \underline{clause} (supplétive) \underline{de transfert différé de propriété} (ou de réserve de propriété) (ex: si le transfert se fait à la fin d'un payement échelonné, tant que tout n'est pas payé, le vendeur reste propriétaire).
			\end{itemize}		
		
			Il est important de connaître le propriétaire pour les risques, par un évènement externe et incertain, non induit par un parti ("la chose périt pour le propriétaire") : la chose est au risque du propriétaire pour autant que le vendeur ne soit pas en faute.
		
			\subsubsection{Délivrance}
			
			Il doit délivrer la chose, avec les accessoires (manuel d'utilisation, certificat d'immatriculation). 
			
			On doit en plus permettre à l'acheteur de contrôler le matériel de voir s'il est confirme à ce qui est dit dans le contrat.
			\pagebreak
				\paragraph{Délai de livraison}		
				Généralement, au niveau du délais de livraison (qui est un accessoire), deux solutions :
		
				\begin{itemize}
					\item on prévoit le délais . C'est une obligation de résultat, on peut prévoir des clauses pénales (montant subit en cas de problèmes) ;
					\item rien n'est mis, mais on prévoit un délai raisonnable. 
				\end{itemize}
		
		
				\paragraph{Conformité}
				
				Il faut vérifier la conformité de la livraison, plusieurs moments :
		
				\begin{itemize}
					\item lorsqu'on en prend possession
					\item dans un certain délai après la livraison, lorsque la conformité ne peut se déceler qu'après certains tests.
				
					La réception sera définitive après ce délai ; cette réception vaut agréation, c'est-à-dire que le vendeur a rempli son obligation de délivrance
				\end{itemize}
				
			
				%[dessin 1]
		
				Il y a 2 types de mise en conformité :
		
				\begin{itemize}
					\item on se réfère aux spécifications techniques
					\item aller au-delà et voir si la machine correspond à l'utilisation (qui est soit définie, soit connue par le vendeur), aux spécifications fonctionnelles.
					
					Pendant ce temps, si ce n'est pas conforme, on peut résoudre la vente ($\neq$ annulation, qui se fait quand le contrat n'est pas encore formé)	
				\end{itemize}
		
				\underline{Clause des 4 coins} : clause qui dit tout ce qui a été convenu, l'intégralité de l'accord est contenue dans le contrat. Cela permet au vendeur de limiter sa responsabilité, si jamais d'autres choses ont été dites verbalement ; la conformité est garantie par ce qu'il y a entre les 4 coins de la page.
				
				\paragraph{Livraison}
		
			Si on ne prévoit rien, pour les choses "certaines", l'endroit de livraison est celui où elles se trouvent, l'acheteur doit se déplacer, sauf s'il y a une clause prévue dans le contrat.
		
		\subsubsection{La garantie}
		
		Le vendeur doit garantir la chose contre les défauts et les vices cachés. 
		
		Conditions d'application :
		
		\begin{itemize}
			\item il y a un vice qui empêche d'utiliser la chose, ou la rend impropre à son usage. Cela concerne les défauts intrinsèques et les défauts fonctionnels. Il faut qu'il soit suffisamment grave pour qu'on s'en plaigne
			\item le défaut doit être caché, c'est-à-dire qui n'aurait pas pu être découvert lors de la livraison
			\item le vice doit exister au moment de la vente
		\end{itemize}
		
		
			\paragraph{Mise en oeuvre de la garantie}
		
			
			Le droit dit qu'on doit agir dans un bref délai. Certains pensent que le délai commence à la livraison, d'autres lorsqu'on se rend compte du défaut. On dispose alors d'un bref délai pour agir et qui varie selon la chose (ex : 6 mois pour un ordi, 3 ans pour une maison), sinon on considère qu'on a accepté le défaut.
		
			Le bref délai peut être reporté s'il y a un accord à l'amiable entre les parties.
		
			\paragraph{Sanction}
			S'il y a un soucis, l'acquéreur peut
			
			\begin{itemize}
				\item soit demander la résolution de la vente et la restitution des biens/de l'argent,
				\item soit garder la chose et on est en partie remboursé.
			\end{itemize}
			
			Si on prouve que le vendeur connaissait le vice, on peut réclamer des dommages et intérêts.
			
			Des clauses peuvent prévoir
		
			\begin{itemize}
				\item des choses en plus, comme des garanties complémentaires. Ou bien un remplacement ou une réparation.
				\item une diminution de la garantie, \underline{clause d'exonération ou limitative de garantie}. Cela ne sera pas pris en compte si le défaut est connu du vendeur et caché à l'acheteur.
			\end{itemize}
		
			Particularité : lorsque c'est un vendeur professionnel, il est présumé de mauvaise foi. Automatiquement, les clauses d'exonération et limitation de garantie ne seront pas effectives. Pour éviter cela, le vendeur doit démontrer son erreur invincible, c'est-à-dire prouver qu'il aurait été impossible de connaître le défaut.
		
			Il y a en plus des règles spéciales qui s'appliquent aux consommateurs (besoins purement privés ; ce qui est dit avant est général), partout en Europe. La garantie est de minimum 2 ans (pas de clause d'exonération de garantie).
		
	
	\subsection{Obligations de l'acheteur}
		
		\subsubsection{Payement}
		L'acheteur a l'obligation de payer le prix. Généralement tout est payé en une fois, ou avec acompte.
	
		En cas de non payement, intérêt de retard (clause pénale, car inexécution prévue à l'avance avec une fixation des dommages).
	
		\subsubsection{Prise de livraison et agréation}
		
	Il a aussi l'obligation prendre livraison et de ne pas accepter si des défauts sont apparents.
	
	
	% Notes 7 - 04 - 2011
	
	\underline{Garantie des vis cachés} : si un défaut apparaît dans un certain délais, l'acheteur peut demander soit la résolution de la vente, soit la diminution, la réfaction du prix (moins-value de la chose). On peut aussi demander le vendeur à remplacer ou réparer la chose (mais on ne peut le forcer).
	
	Pour faire valoir cette garantie, le défaut doit être relativement grave (empêchant l'utilisation de la chose) et caché (et révélé plus tard, non décelé lors de l'examen). Il faut le faire sous un certain délais, sinon on considère que l'on a accepté tacitement le défaut.
	
	On peut demander des dommages et intérêts, il faut prouver le dommage et la mauvaise foi du vendeur, c'est-à-dire prouver qu'il connaissait le vis et qu'il ne l'a pas révélé. Si on achète chez un professionnel, on le considère d'office comme de mauvaise foi. Le vendeur devra prouver son ignorance invincible, c'est-à-dire qu'il n'aurait jamais su le savoir.
	
	Si l'acheteur est professionnel, on peut supposer qu'il aurait du trouver le vice lors de la période d'agrégation.
	
	Le vendeur peut s'exonérer de sa garantie (ne s'applique pas s'il est de mauvaise foi).
	
	Lorsque l'acheteur est un consommateur (pour ses besoins privés), il y a des mécanismes de garantie encore plus protecteurs (garantie de 2 ans minimum.
	
	S'il y a agréation, tout recours est impossible contre le fournisseur.
	
	
	
	\section{Contrat software}
	
	Il faut distinguer les progiciels (destinés à un grand public ou à répondre aux besoins de plusieurs utilisateurs confrontés à un même type d'application ; gestion de stock, comptabilité, etc) et les logiciels sur mesure (pour un besoin spécifique d'un utilisateur déterminé).
	
	Si c'est un progiciel, ce n'est pas une vente, car on ne donne pas de droits de propriété, seulement un droit d'utilisation (licence d'utilisation).
	
	Si c'est un logiciel sur mesure, c'est une prestation de service.
	
		\subsection{Droits d'utilisation}
		
		Pour un progiciel,
		\begin{itemize}
			\item le droit concédé est non-exclusif, c'est-à-dire qu'on n'est pas le seul à utiliser le même programme;
			\item personnel ; il est interdit de céder à un tiers le bénéfice du contrat ;
			\item ne peut être utilisé que dans un environnement déterminé (OS, ...) ;
			\item le code source ne sera jamais fourni à l'utilisateur.
		\end{itemize}
		
		Pour un logiciel sur mesure, 
		
		\begin{itemize}
			\item droit exclusif ou non (selon les accords convenus);
			\item personnel ou non ; 
			\item pour le code source, ça dépend de ce qui est prévu. Si le client paye tout et si rien n'est dit, on considère que les droits sont cédés. On peut céder les droits de l'oeuvre au client, ou bien on cède une licence d'utilisation, ou bien on peut réutiliser l'oeuvre pour d'autres clients, avec ou sans royalties pour le client original (car c'est lui qui a financé le développement du logiciel).
			
			Si le fournisseur garde le code source, il faut que le client puisse toujours y avoir accès.
			
			Généralement, au début on délivre le code source chez un notaire ou quelqu'un de confiance, qui peut délivrer le code à l'utilisateur au cas où il doit y accéder.			
			
		\end{itemize}
	
	
		\subsection{Conformité}
		
		Idem que pour le matériel, si ce n'est que les périodes de tests sont généralement plus longues.
		
		\subsection{Garantie}
		
		Idem qu'en hardware, mais les défauts concernent le software, soit les erreurs de programmation ou de conception.
		
		Deux types de garantis entrent en jeu :
		\begin{itemize}
			\item la garantie des vice cachés
		
			\item la garantie d'éviction. Elle est enclenchée lors de plagiat du vendeur ; le vendeur doit garantir le client contre l'éviction d'un tiers, c'est-à-dire qu'un tiers ne viendra pas directement interdire le client d'utiliser le logiciel.
		\end{itemize}
		
		En cas d'attaque par un tiers,
		
		\begin{enumerate}
			\item demander l'aide du fournisseur dans la défense
			\item si le tiers triomphe contre l'utilisateur, il faut mettre en jeu la garantie : résolution du contrat et dommages et intérêts.
		\end{enumerate}
		
		\subsection{Modification}
		
		Le plus souvent, un client peut modifier le logiciel sous licence.
		
	\section{Contrat de maintenance}
	
	Il y a deux types de maintenance : maintenance préventive (vérification) ou curative (un problème s'est produit).
	
	
	Il ne faut pas oublier de mettre un délai d'intervention dans le contrat. Si ce délai n'est pas respecté, ou si la réparation dure longtemps, on prévoit généralement une obligation à l'informaticien de prévoir du matériel de remplacement.
	
	Généralement, le fournisseur limite sa responsabilité (ex : si le matériel a été abîmé par le client).
	
	La durée du contrat est déterminée dans le contrat, avec une tacite reconduction (tant que la relation se poursuit, que l'informaticien est payé, on considère le contrat renouvelé pour la durée prévue dans le contrat). Si on veut changer, il faut attendre la fin de la reconduction.
	
	On peut prévoir une reconduction à durée indéterminée. Si on veut y mettre un terme, il faut donner un préavis, et on peut finir avant la période (renouvelée) par le contrat.
	
		\subsection{Garantie de disponibilité}
		
		On garantit que la machine fonctionnera bien pendant une certaine durée. Si cette durée n'est pas respectée, dommage et intérêts.
	
	
	\section{Contrat de conseil informatique}
	
	Contrat de prestation de service.
	
	
	
	\section{Contrat de sous-traitance}
	
	%3 grands principes :
	
	%1) On n'a de relation qu'avec la personne avec qui on a passé un contrat. Ainsi, le client n'a pas d'action directe avec le sous-traitant d'un fournisseur.
	%2, 3) ...
		
	 On n'a de relation qu'avec la personne avec qui on a passé un contrat. Ainsi, le client n'a pas d'action directe avec le sous-traitant d'un fournisseur. Le problème est que si le fournisseur disparaît, aucun recours n'est possible.
		
	Par contre, si le sous-traitant n'est pas payé par le fournisseur, il peut directement réclamer au client final, pour autant que le client n'a pas payé.
	
		\subsection{Out-sourcing}	
		Il y a out-sourcing lorsqu'une entreprise délègue en permanence à une autre entreprise la gestion d'un département qui n'est pas essentiel (pas dans le core business, la prestation principale).		
		
		Cela permet aux entreprises de payer une autre entreprise, et non des travailleurs (ce qui coûte plus cher).	
		
		Par rapport à la sous-traitance,  1) le transfert est permanent et  2) on fournit un service à l'entreprise pour l'entreprise, alors que dans la sous-traitance on agit sur le core business. De plus, 3) la sous-traitance est une prestation au profit d'un tiers (le client), alors que le profit est pour l'entreprise dans l'out-sourcing.
		
		Clauses d'out-sourcing :
	
		\begin{itemize}
			\item clause de benchmarking : on veut garantir que le système informatique utilisé par l'entreprise est performant ;
			\item clause de reporting : demande de rapports pour organiser les relations entre les deux entreprises (réunion, rapports, etc) ;
			\item clause de réversibilité : possibilité de changer d'entreprise ; retour à l'ancien système. Il y a obligation pour l'informaticien de passer le flambeau à ceux qui vont reprendre le travail (formation, etc).
		\end{itemize}
	
	
	
	